\documentclass[letta4 paper]{article}
% Set target color model to RGB
\usepackage[inner=2.0cm,outer=2.0cm,top=2.5cm,bottom=2.5cm]{geometry}
\usepackage{setspace}
\usepackage[rgb]{xcolor}
\usepackage{verbatim}
\usepackage{subcaption}
\usepackage{amsgen,amsmath,amstext,amsbsy,amsopn,tikz,amssymb}%,tkz-linknodes}
\usepackage{fancyhdr}
\usepackage[colorlinks=true, urlcolor=blue,  linkcolor=blue, citecolor=blue]{hyperref}
\usepackage[colorinlistoftodos]{todonotes}
\usepackage{rotating}
\usepackage{listings}
\usepackage{amsmath,amsfonts,amssymb}
\usepackage{amsmath}
\lstset{
%	language=bash,
basicstyle=\ttfamily
}
\usepackage{mathtools}
\DeclarePairedDelimiter{\norm}{\lVert}{\rVert} 
\usepackage{microtype}

\newcommand{\ra}[1]{\renewcommand{\arraystretch}{#1}}

\newtheorem{thm}{Theorem}[section]
\newtheorem{prop}[thm]{Proposition}
\newtheorem{lem}[thm]{Lemma}
\newtheorem{cor}[thm]{Corollary}
\newtheorem{defn}[thm]{Definition}
\newtheorem{rem}[thm]{Remark}
\numberwithin{equation}{section}
\graphicspath{ {./img/} }

\newcommand{\homework}[6]{
\pagestyle{myheadings}
\thispagestyle{plain}
\newpage
\setcounter{page}{1}
\noindent
\begin{center}
	\framebox{
		\vbox{\vspace{2mm}
			\hbox to 6.28in { {\bf F1TENTH Autonomous Racing \hfill {\small (#2)}} }
			\vspace{6mm}
			\hbox to 6.28in { {\Large \hfill #1  \hfill} }
			\vspace{6mm}
			\hbox to 6.28in { {\it Instructor: {\rm #3} \hfill Name: {\rm #5}, StudentID: {\rm #6}} }
			%\hbox to 6.28in { {\it T\textbf{A:} #4  \hfill #6}}
			\vspace{2mm}}
	}
\end{center}
\markboth{#5 -- #1}{#5 -- #1}
\vspace*{4mm}
}


\newcommand{\problem}[3]{~\\\fbox{\textbf{Problem #1: #2}}\hfill (#3 points)\newline}
\newcommand{\subproblem}[1]{~\newline\textbf{(#1)}}
\newcommand{\D}{\mathcal{D}}
\newcommand{\Hy}{\mathcal{H}}
\newcommand{\VS}{\textrm{VS}}
\newcommand{\solution}{~\newline\textbf{\textit{(Solution)}} }

\newcommand{\bbF}{\mathbb{F}}
\newcommand{\bbX}{\mathbb{X}}
\newcommand{\bI}{\mathbf{I}}
\newcommand{\bX}{\mathbf{X}}
\newcommand{\bY}{\mathbf{Y}}
\newcommand{\bepsilon}{\boldsymbol{\epsilon}}
\newcommand{\balpha}{\boldsymbol{\alpha}}
\newcommand{\bbeta}{\boldsymbol{\beta}}
\newcommand{\0}{\mathbf{0}}

\DeclareMathOperator*{\argmax}{arg\,max}
\DeclareMathOperator*{\argmin}{arg\,min}


\usepackage{booktabs}



\begin{document}

\homework {Lab 5: Scan Matching}{Due Date: 2022-02-21}{Rahul Mangharam}{}{Team 5}{Various}
\thispagestyle{empty}
% -------- DO NOT REMOVE THIS LICENSE PARAGRAPH	----------------%
\begin{table}[h]
	\begin{tabular}{l p{14cm}}
		\raisebox{-2cm}{\includegraphics[scale=0.5, height=2.5cm]{f1_stickers_03} } & \textit{This lab and all related course material on \href{http://f1tenth.org/}{F1TENTH Autonomous Racing} has been developed by the Safe Autonomous Systems Lab at the University of Pennsylvania (Dr. Rahul Mangharam). It is licensed under a \href{https://creativecommons.org/licenses/by-nc-sa/4.0/}{Creative Commons Attribution-NonCommercial-ShareAlike 4.0 International License.} You may download, use, and modify the material, but must give attribution appropriately. Best practices can be found \href{https://wiki.creativecommons.org/wiki/best_practices_for_attribution}{here}.}
	\end{tabular}
\end{table}
% -------- DO NOT REMOVE THIS LICENSE PARAGRAPH	----------------%

\noindent \large{\textbf{Course Policy:}} Read all the instructions below carefully before you start working on the assignment, and before you make a submission. All sources of material must be cited. The University Academic Code of Conduct will be strictly enforced.
\\
\\
\textbf{THIS IS A GROUP ASSIGNMENT}. Submit one from each team.\\

\section{Theoretical Questions}
\begin{enumerate}
	\item $M_{i}=\left(\begin{array}{cccc}{1} & {0} & {p_{i 0}} & {-p_{i 1}} \\ {0} & {1} & {p_{i 1}} & {p_{i 0}}\end{array}\right)$
	\begin{enumerate}
		\item Show that $B_{i} \coloneqq M_{i}^{T} M_{i}$ is symmetric. \\
		\textbf{Solution:}
		\begin{align}
			B_i^T &= (M_i^T M_i)^T \\
			&= M_i^T (M_i^T)^T \\
			&= M_i^T M_i=B_i \\
			\Aboxed{B_i^T &= B_i}
		\end{align}
		
		
		\item Demonstrate that $B_{i}$ is positive semi-definite\\
		\textbf{Solution:}
		For any $x\in \mathbb{R}^4$,
		\begin{align}
			x^T B_i x &= x^T M_i^T M_i x \\
			&= (M_i x)^T (M_i x) \\
			&= \norm{M_i x}_2^2 \geqslant 0 \\
			x^T B_i x &\geqslant 0\\
			\Aboxed{\Rightarrow B_i &\succeq 0}
		\end{align}
		
	\end{enumerate}
	
	\item The following is the optimization problem:
	\[ 
	x^{*}=\argmin_{x \in \mathbb{R}^{4}} \sum_{i=1}^{n}\left\|M_{i} x-\pi_{i}\right\|_{2}^{2} \quad \text { s.t. } \quad x_{3}^{2}+x_{4}^{2}=1
	\] 
	\begin{enumerate}
		\item Find the matrices $M$, $W$ and $g$ which give you the formulation 
		\[
		x^{*}=\argmin_{x \in \mathbb{R}^{4}} x^{T} M x+g^{T} x 
		\quad \text { s.t. } x^{T} W x=1
		\]
		\textbf{Solution:}
		\begin{align}
			x^{*}&=\argmin_{x \in \mathbb{R}^{4}} \sum_{i=1}^{n}\norm{M_{i} x-\pi_{i}}_{2}^{2} \\
			&= \argmin_{x \in \mathbb{R}^{4}}
			\sum_{i=1}^{n} \left(x^T M_i^T M_i x - 2\pi_i^T M_i x + \pi_i^T \pi_i\right) \\
			&= \argmin_{x \in \mathbb{R}^{4}}
			\sum_{i=1}^{n} \left(x^T M_i^T M_i x - 2\pi_i^T M_i x\right)\\
			&= \argmin_{x \in \mathbb{R}^{4}}
			x^T \left(\sum_{i=1}^{n}M_i^T M_i\right) x - 2\left(\sum_{i=1}^{n}\pi_i^T M_i\right) x \\
			\Aboxed{\Rightarrow M &= \sum_{i=1}^{n}M_i^T M_i \quad g = 2\sum_{i=1}^{n}M_i ^T \pi_i}
		\end{align}
		\begin{align}
			x_{3}^{2}+x_{4}^{2} 
			&=
			x^T 
			\begin{bmatrix}
				0 & 0 & 0 & 0 \\
				0 & 0 & 0 & 0 \\
				0 & 0 & 1 & 0 \\
				0 & 0 & 0 & 1
			\end{bmatrix}
			x = 1 \\
			\Aboxed{\Rightarrow W &= 
				{\begin{bmatrix}
						0 & 0 & 0 & 0 \\
						0 & 0 & 0 & 0 \\
						0 & 0 & 1 & 0 \\
						0 & 0 & 0 & 1
			\end{bmatrix}}}
		\end{align}
		\\		
		
		\item Show that $M$ and $W$ are positive semi definite.
		\textbf{Solution:}
		\\
		For any $x = 
		\begin{bmatrix}
			x_1 & x_2 & x_3 & x_4         
		\end{bmatrix}^T \in \mathbb{R}^4$,
		\begin{align}
			x^T M x
			&=
			x^T \sum_{i=1}^{n}M_i^T M_i x\\
			&=
			\sum_{i=1}^{n} (x^T M_i^T M_i x) \\
			&=
			\sum_{i=1}^{n} (M_i x)^T (M_i x) \\
			&=
			\sum_{i=1}^{n} \norm{M_i x}_2^2 
			\geqslant
			0 \\
			\Aboxed{\Rightarrow M &\succeq 0}
		\end{align}
		\begin{align}
			x^T W x &= x_3^2 + x_4^2 \geqslant 0 \\ 
			\Aboxed{\Rightarrow W &\succeq 0}
		\end{align}
		\\		
		
	\end{enumerate}
	
\end{enumerate}

\section{Approach and Performance}

\subsection{Approach}

We implemented the point-to-line iterative closest point algorithm as shown in the course slides.

\subsection{Performance}

\href{https://youtu.be/OCojTZnd3rk}{Associated video.}

The implementation works fairly well. It rejects a pose reset without too much error, though that might be based on the implementation of the 2D Pose Estimate function in rviz. 

As for mid-driving performance (beginning at \href{https://youtu.be/OCojTZnd3rk?t=29}{0:29} in the linked video), the algorithm works best if there are minimal changes to the scan's readings. For example, when it drives straight down the hallway, the scan's changes are fairly simple, so there is not much lifting for the optimization program to do. Even the turn at \href{https://youtu.be/OCojTZnd3rk?t=38}{0:38} is handled nicely. 

During a teleoperation-related jitter at \href{https://youtu.be/OCojTZnd3rk?t=42}{0:42}, however, the pose estimation acquires some drift. In order to emphasize this effect, we swerve the car back and forth at \href{https://youtu.be/OCojTZnd3rk?t=59}{0:59}, creating even more drift. Then, during the turn at \href{https://youtu.be/OCojTZnd3rk?t=74}{1:14}, where the wall geometry is more complex, even more drift is acquired as the optimizer struggles to keep up.

These issues appear to stem from code optimization as opposed to fundamental misunderstandings.







\end{document} 
